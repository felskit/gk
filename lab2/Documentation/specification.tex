\documentclass[12pt]{article}

\usepackage[in]{fullpage}
\usepackage{fancyhdr}

%\usepackage{amsmath}
%\usepackage{amssymb}
%\usepackage{mathtools}
%\usepackage{graphicx}

\usepackage[polish]{babel}
\usepackage[T1]{fontenc}
\usepackage[utf8]{inputenc}

\renewcommand{\thesection}{\arabic{section}.}
%\renewcommand*{\arraystretch}{1.25}

\begin{document}

\begin{titlepage}

  \fancyhead{}
  \fancyfoot{}

  \fancypagestyle{empty}{
      \fancyhead[R]{\large{16.11.2016}}
  }

  \vspace*{\stretch{1.0}}
  \begin{center}
    \huge{GRAFIKA KOMPUTEROWA 1}\\
    \huge{PROJEKT LABORATORYJNY 2}\\
  \vspace*{\stretch{0.2}}
    \Large{SPECYFIKACJA}\\
  \vspace*{\stretch{0.2}}
    \Large{EDYTOR WIELOKĄTÓW - WYPEŁNIANIE I OBCINANIE}\\
  \vspace*{\stretch{0.2}}
    \large\textbf{TYMON FELSKI, C1}\\
  \end{center}
    \vspace*{\stretch{1.0}}   

\end{titlepage}

\section{Instrukcja obsługi programu}
\textbf{Dodawanie wielokąta}\\Wielokąty można dodawać klikając \textbf{LPM}. Na ekranie będą pojawiać się kolejne wierzchołki. Rysowanie zakończy się, jeżeli łamana zostanie zamknięta (postawimy ostatni wierzchołek dostatecznie blisko początkowego). Rysowanie można anulować wciskając \textbf{Escape}.\\[\baselineskip]
\textbf{Przesuwanie wielokąta}\\Wielokąt można przesunąć wciskając \textbf{Shift + LPM} lub \textbf{ŚPM}, gdy kursor znajduje się w środku tego wielokąta. Jeżeli kilka wielokątów się pokrywa, o wyborze decydować będzie kolejność ich rysowania.\\[\baselineskip]
\textbf{Przesuwanie wierzchołka}\\Wierzchołek w narysowanym wielokącie można przesunąć przytrzymując na nim \textbf{LPM}.\\[\baselineskip]
\textbf{Usuwanie wierzchołka}\\Wierzchołek w narysowanym wielokącie można usunąć wciskając na nim \textbf{PPM} i wybierając opcję \textbf{Delete} w menu kontekstowym. Wierzchołki można usuwać, jeżeli ich liczba w wielokącie jest większa od 3, w przeciwnym wypadku zostanie wyświetlony komunikat o błędzie.\\[\baselineskip]
\textbf{Dzielenie krawędzi}\\Krawędź w narysowanym wielokącie można podzielić na dwie, dodając w środku wierzchołek. W tym celu należy wcisnąć na niej \textbf{PPM} i wybrać opcję \textbf{Split} w menu kontekstowym.\\[\baselineskip]
\textbf{Wypełnianie wielokąta}\\Wielokąt można wypełnić wciskając na nim \textbf{PPM} i wybierając opcję \textbf{Filled}. Wielokąt wypełni się domyślnym kolorem, jeżeli w aplikacji nie zostało ustawione tło.\\[\baselineskip]
\textbf{Ustawianie tła wielokątów}\\Aby załadowac tło wielokątów, należy wcisnąć przycisk \textbf{Load} w sekcji \textbf{Background} na bocznym panelu. Otworzy się okno dialogowe, w którym należy wybrać plik w rozszerzeniu *.png, *.bmp, *.jpg lub *.jpeg.\\[\baselineskip]
\textbf{Ustawianie mapy wysokości (zaburzeń)}\\Aby załadować mapę wysokości wypełnienia, należy wcisnąć przycisk \textbf{Load} w sekcji \textbf{Heightmap} na bocznym panelu. Otworzy się okno dialogowe, w którym należy wybrać plik w rozszerzeniu *.png, *.bmp, *.jpg lub *.jpeg.\\[\baselineskip]
\textbf{Usuwanie tła lub mapy wysokości}\\Zarówno tło wielokątów, jak i mapę wysokości można usunąć po wczytaniu, aby powrócić do ustawień domyślnych. Wystarczy wcisnąć przycisk \textbf{Remove} na bocznym panelu w sekcjach odpowiednio \textbf{Background} oraz \textbf{Heightmap}.\\[\baselineskip]
\textbf{Zmiana funkcji wektorów normalnych}\\Na bocznym panelu w sekcji \textbf{Normal vectors} znajdują się radio buttony służące do zmiany predefiniowanych funkcji wektorów normalnych.\\[\baselineskip]
\textbf{Zmiana rodzaju światła}\\W sekcji \textbf{Light} panelu bocznego można przełączać się pomiędzy opcjami światła \textbf{Static infinity} oraz \textbf{Dynamic}.\\[\baselineskip]
\textbf{Zmiana koloru światła}\\Znajdujący się w panelu bocznym przycisk z kolorem służy do otwierania okna wyboru koloru. Kolor przycisku jest obecnym kolor światła (domyślnie biały).\\[\baselineskip]
\textbf{Poruszanie światłem}\\W sekcji \textbf{Light} panelu bocznego znajduje się pięć textboxów, w których można zmieniać odpowiednie parametry dynamicznego oświetlenia (x środka, y środka, wysokość, promień oraz okres obiegu). W przypadku podania nieprawidłowych danych, zostaną zresetowane do domyślnych.\\[\baselineskip]
\textbf{Zaznaczanie wielokątów}\\Wielokąty można zaznaczać wciskając \textbf{Ctrl + LPM} lub wciskając na nich \textbf{PPM} i wybierając opcję \textbf{Selected}. Zaznaczony wielokąt podświetli się na czerwono. Ponowne zaznaczenie zaznaczonego wielokąta spowoduje jego odznaczenie. Można mieć zaznaczone maksymalnie dwa wielokąty. Wciskając \textbf{Ctrl + Shift + A} można odznaczyć wszystkie zanzaczone wielokąty.\\[\baselineskip]
\textbf{Suma wielokątów}\\Mając zaznaczone dwa nierozłączne wielokąty bez samoprzecięć, można wcisnąć \textbf{PPM} na dowolnym z nich i wybrać opcję \textbf{Union}. Zaznaczone wielokąty zostaną usunięte, a zastąpi je jeden wielokąt, będący ich sumą. Jeżeli ww. warunki nie będą spełnione, zostanie wyświetlony komunikat o błędzie.

\end{document}